\documentclass[letterpaper,11pt]{article}

% Choose bibliography style for formatting list of publications
\usepackage[style=ieee,url=false,doi=false,maxbibnames=99,sorting=ydnt,dashed=false]{biblatex}

% Choose theme, e.g. black, RedViolet, ForestGreen, MidnightBlue
\def\theme{MidnightBlue}

\usepackage{simplecv}

\boldname{Florian}{Valade}{N.}

\begin{document}

% Heading
\headinginline{Valade Florian}{
    Email: \email{florian\_val@outlook.fr} \\
    LinkedIn: \linkedin{florian-valade} \\
    GitHub: \github{FlorianVal} \\
    Site Web: \website{fvalade.fr}
}

% Education Section
\section{Education}

\outerlist{
    \entrybig
    {\textbf{Université Gustave Eiffel}}{Paris, France}
    {Thèse : Généralisation d’algorithmes légers pour la reconnaissance sur caméra embarquée}{2022\textendash Présent}

    \entrybig
    {\textbf{ECE Paris}}{Paris, France}
    {Diplôme d'Ingénieur en Systèmes d'Information, Big Data et Machine Learning}{2021}

    \entrybig
    {\textbf{Lycée L'Espérance}}{Paris, France}
    {Baccalauréat Scientifique}{2015}
}

% Publications Section
\section{Publications}

\outerlist{
    \entrybig
    {EERO: Early Exit with Reject Option (2025)}{}
    {- Recherche sur les techniques statistiques pour optimiser les mécanismes de sortie anticipée dans les tâches de classification. Accepté à UAI 2025}{}

    \entrybig
    {Accélération de l'inférence des grands modèles de langage avec des sorties anticipées auto-supervisées (2024)}{}
    {- Extension des techniques de sortie anticipée aux grands modèles de langage pour l'optimisation de l'inférence}{}
}

% Experience Section
\section{Experience}

\outerlist{
    \entrybig
    {\textbf{Fujitsu - Université Gustave Eiffel}}{Paris, France}
    {Thèse CIFRE, Ingénieur de recherche}{2022\textendash Présent}
    \innerlist{
        \entry{Sujet: Généralisation d’algorithmes légers pour la reconnaissance sur caméra embarquée.}
    }

    \entrybig
    {\textbf{Fujitsu}}{Paris, France}
    {Data Scientist}{2021\textendash 2022}
    \innerlist{
        \entry{Développement et gestion de projets en vision par ordinateur et deep learning.}
    }

    \entrybig
    {\textbf{Fujitsu - ECE Paris}}{Paris, France}
    {Apprenti Data Scientist, spécialisé en Computer Vision}{2018\textendash 2021}
    \innerlist{
        \entry{Formation et application de techniques de vision par ordinateur et machine learning.}
    }

    \entrybig
    {\textbf{Fujitsu}}{Paris, France}
    {Stagiaire Data Scientist}{Avril 2018\textendash Sept. 2018}
    \innerlist{
        \entry{Développement de démonstrations en deep learning.}
    }
}

% Skills Section
\sidebyside
    {
        \section{Compétences}

        \outerlist{
            \entry{\textbf{Langages de programmation:} Python, Java, C\#, C, SQL}
            \entry{\textbf{Frameworks et Outils:} Pytorch, Tensorflow, Docker, Git}
            \entry{\textbf{Développement et Systèmes:} Front End avec React, DevOps, Réseau, Calcul Distribué, Cyber Sécurité, Administration Système}
        }
    }
    {
        \section{Langues}
        \denseouterlist{
            \entry{\textbf{Anglais: } Courant}
            \entry{\textbf{Français: } Natif}
            \entry{\textbf{Espagnol: } Intermédiaire}
        }
    }

% Projects Section
\section{Projets}

\vspace{1em}

\begin{minipage}[t]{0.505\textwidth}

\outerlist{

\entrybig[\textbullet]
{Handterpret (Tensorflow et électronique, 2020)}{}
{Chef de projet de fin détude en école d'ingénieur }{}
{Détection de la position de la main grâce a des capteurs infrarouge sur le poignet}

}

\end{minipage}
\begin{minipage}[t]{0.48\textwidth}

\outerlist{

\entrybig[\textbullet]
{AutoCradle (Tensorflow et électronique, 2017)}{}
{Projet d'équipe}{}
{Détection automatique de pleurs de bébé pour activer le balancement du landau}

}

\end{minipage}

\end{document}
