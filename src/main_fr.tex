\documentclass[letterpaper,11pt]{article}

% Choose bibliography style for formatting list of publications
\usepackage[style=ieee,url=false,doi=false,maxbibnames=99,sorting=ydnt,dashed=false]{biblatex}

% Choose theme, e.g. black, RedViolet, ForestGreen, MidnightBlue
\def\theme{MidnightBlue}

\usepackage{simplecv}

\boldname{Florian}{Valade}{N.}

\begin{document}

% En-tête
\headinginline{Valade Florian}{
    \begin{tabular}{l@{\hspace{1em}}l}
        Email:     & \email{florian\_val@outlook.fr} \\
        Téléphone: & +33 6 17 57 19 12               \\
        LinkedIn:  & \linkedin{florian-valade}       \\
        GitHub:    & \github{FlorianVal}             \\
        Site web:  & \website{fvalade.fr}
    \end{tabular}
}

{\normalfont{À la recherche d'un poste de recherche}}

% Section Formation
\section{Formation}

\outerlist{
    \entrybig%
    {\textbf{Université Gustave Eiffel}}{Paris, France}
    {Doctorat en efficacité des algorithmes d'apprentissage profond}{2022 \textendash{} Présent}
    
    \entrybig%
    {\textbf{ECE Paris}}{Paris, France}
    {Master en informatique, Big Data et apprentissage automatique}{2021}
    
    \entrybig%
    {\textbf{Lycée L'Espérance}}{Paris, France}
    {Baccalauréat scientifique}{2015}
}

% Section Publications
\section{Publications}

\outerlist{
    \entrybig%
    {EERO: Early Exit with Reject Option (2025)}{}
    {- Optimisation des mécanismes de sortie anticipée dans les tâches de classification.}{UAI 2025}
    
    \entrybig%
    {Accelerating Large Language Model Inference with Self-Supervised Early Exits (2024)}{}
    {- Extension des techniques de sortie anticipée aux grands modèles de langage pour optimiser l'inférence.}{}
}

% Section Expérience
\section{Expérience}

\outerlist{
    \entrybig
    {\textbf{Fujitsu -- Université Gustave Eiffel}}{Paris, France}
    {Doctorant, Ingénieur de recherche}{2022 \textendash{} Présent}
    \innerlist{
        \entry{Optimisation d'algorithmes d'apprentissage profond pour la reconnaissance sur caméras embarquées à l'aide de techniques statistiques sur les sorties anticipées (Early Exit).}
        \entry{Entraînement et ajustement (fine-tuning) de grands modèles de langage.}
        \entry{Création de plusieurs projets internes pour améliorer les workflows des collaborateurs via LLMs, fine-tuning, vision par ordinateur et curation/validation de données.}
        \entry{Gestion de serveurs multi-GPU pour l'entraînement et l'inférence.}
        \entry{Enseignement de la Vision par Ordinateur et du Traitement Automatique du Langage (NLP) à des étudiants de Master 2.}
    }
    
    \entrybig%
    {\textbf{Fujitsu}}{Paris, France}
    {Data Scientist}{2021 \textendash{} 2022}
    \innerlist{
        \entry{Développement et gestion de projets en vision par ordinateur et apprentissage profond.}
    }
    
    \entrybig%
    {\textbf{Fujitsu -- ECE Paris}}{Paris, France}
    {Apprenti Data Scientist, spécialisé en vision par ordinateur}{2018 \textendash{} 2021}
    \innerlist{
        \entry{Mise en œuvre de techniques de vision par ordinateur et d'apprentissage automatique.}
    }
    
    \entrybig%
    {\textbf{Fujitsu}}{Paris, France}
    {Stagiaire Data Scientist}{Avril 2018 \textendash{} Sept. 2018}
    \innerlist{
        \entry{Développement de démonstrations en apprentissage profond.}
    }
}

% Section Compétences
\section{Compétences}

\outerlist{
    \entry{\textbf{Langages de programmation:} Python, Java, C\#, C, SQL}
    \entry{\textbf{Frameworks et outils:} PyTorch, TensorFlow, Docker, Git, JAX, MLX}
    \entry{\textbf{Développement et systèmes:} Front-end avec React, DevOps, Réseau, Calcul distribué, Cybersécurité, Administration système}
    \entry{\textbf{Langues:} Anglais (Courant), Français (Natif), Espagnol (Intermédiaire)}
    \entry{\textbf{Compétences transversales:} Curiosité, Résolution de problèmes, Communication, Collaboration, Travail d'équipe}
}

% Section Projets
\section{Projets}

\outerlist{
    \entrybig[\textbullet]%
    {FreshDetect (PyTorch et Docker, 2022)}{}
    {Data scientist}{}
    {Développé une solution de bout en bout pour la classification en temps réel de fruits et légumes en supermarché à l'aide de l'apprentissage profond. Intégration avec les systèmes du magasin via des microservices conteneurisés.}
    
    \entrybig[\textbullet]%
    {Handterpret (TensorFlow et électronique, 2020)}{}
    {Chef de projet de fin d'études}{}
    {Détection de la position de la main grâce à des capteurs infrarouges au poignet.}
    
    \entrybig[\textbullet]%
    {AutoCradle (TensorFlow et électronique, 2017)}{}
    {Projet d'équipe}{}
    {Mise en œuvre de la détection automatique des pleurs de bébé pour activer le balancement du berceau.}
}

\end{document}
